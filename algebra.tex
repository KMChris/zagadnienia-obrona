\subsection{Wyznacznik macierzy}
\subsubsection*{Definicja:}
Wyznacznik to liczba przypisana macierzy kwadratowej, oznaczana np. $\det(A)$, wykorzystywana m.in. do badania odwracalności macierzy. Macierz jest odwracalna wtedy i tylko wtedy, gdy jej wyznacznik jest różny od zera.

\subsubsection*{Metody obliczania:}
\begin{itemize}
    \item \textbf{Rozwinięcie Laplace'a:} rozwinięcie względem dowolnego wiersza lub kolumny przy użyciu mniejszych wyznaczników (minorów). Dobrze nadaje się dla małych macierzy.
    \item \textbf{Eliminacja Gaussa:} przekształcamy macierz do postaci trójkątnej; wyznacznik to iloczyn elementów na przekątnej, z uwzględnieniem znaków wynikających z zamian wierszy.
\end{itemize}

\subsubsection*{Interpretacja geometryczna:}
Dla macierzy $2 \times 2$ lub $3 \times 3$, wyznacznik odpowiada odpowiednio polu powierzchni lub objętości równoległoboku/równoległościanu rozpiętego przez kolumny macierzy. Znak wyznacznika informuje o orientacji układu (np. dodatni -- zachowana orientacja).

\subsection{Wartości i wektory własne}
\subsubsection*{Definicja:}
Dla macierzy kwadratowej $A$, liczba $\lambda$ jest \textbf{wartością własną}, jeśli istnieje niezerowy wektor $v$, taki że:

$$
A v = \lambda v
$$

Wektor $v$ nazywany jest \textbf{wektorem własnym} odpowiadającym wartości $\lambda$.

\subsubsection*{Zastosowania:}
\begin{itemize}
    \item \textbf{Diagonalizacja macierzy:} jeśli macierz $A$ ma $n$ liniowo niezależnych wektorów własnych, to można ją przedstawić w postaci $A = PDP^{-1}$, gdzie $D$ to macierz diagonalna z wartościami własnymi. Ułatwia to np. potęgowanie macierzy.
    \item \textbf{Zastosowania praktyczne:} analiza stabilności układów dynamicznych, PCA (analiza głównych składowych), mechanika kwantowa, przetwarzanie obrazów, grafy.
\end{itemize}

\subsection{Układy równań liniowych}
\subsubsection*{Liczba rozwiązań:}
Układ równań liniowych może mieć:
\begin{itemize}
    \item jedno rozwiązanie (układ oznaczony),
    \item nieskończenie wiele rozwiązań (układ nieoznaczony),
    \item brak rozwiązań (układ sprzeczny).
\end{itemize}

\subsubsection*{Twierdzenie Kroneckera-Capellego:}
Układ równań liniowych ma rozwiązanie wtedy i tylko wtedy, gdy rząd macierzy głównej $A$ jest równy rzędowi macierzy rozszerzonej $(A|b)$.
\begin{itemize}
    \item Jeśli ten rząd jest równy liczbie niewiadomych $\rightarrow$ układ oznaczony.
    \item Jeśli mniejszy $\rightarrow$ nieskończenie wiele rozwiązań.
    \item Jeśli różny $\rightarrow$ układ sprzeczny.
\end{itemize}
