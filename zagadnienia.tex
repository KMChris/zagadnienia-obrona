\documentclass[12pt]{article}
\usepackage[utf8]{inputenc}
\usepackage[polish]{babel}
\usepackage[T1]{fontenc}
\usepackage{amsmath, amssymb}
\usepackage{mathtools}
\usepackage{enumitem}
\usepackage{geometry}
\usepackage{lmodern}
\usepackage{titlesec}
\usepackage{graphicx}
\usepackage{float}
\usepackage[hidelinks]{hyperref}

\geometry{
  a4paper,
  left=25mm,
  right=25mm,
  top=25mm,
  bottom=25mm
}
\setcounter{secnumdepth}{2}
\titleformat{\section}{\large\bfseries}{\thesection.}{1em}{}
\setlength{\parindent}{0pt}
\setlength{\parskip}{0.8em}
\renewcommand\thesection{}
\renewcommand\thesubsection{\arabic{subsection}.}
\titleformat{\subsection}
  {\normalfont\small\bfseries}
  {\thesubsection}{1em}{}

\newcommand{\xrightarrowd}{\xrightarrow{\text{d}}}


\title{\bfseries Zagadnienia na obronę}
\author{Matematyka}
\date{\today}

\begin{document}
\maketitle
\tableofcontents

\section*{Analiza matematyczna}
\addcontentsline{toc}{section}{Analiza matematyczna}
\subsection{Centralne Twierdzenie Graniczne (CTG)}

\subsubsection*{Definicja:}
Centralne Twierdzenie Graniczne mówi, że jeśli $X_1, X_2, ..., X_n$ są niezależnymi, identycznie rozłożonymi zmiennymi losowymi o skończonej wartości oczekiwanej $\mu$ i skończonej wariancji $\sigma^2$, to suma (lub średnia) tych zmiennych po odpowiednim przeskalowaniu dąży rozkładem do rozkładu normalnego, gdy $n \to \infty$.

Formalnie:

$$
\frac{\sum_{i=1}^n X_i - n\mu}{\sigma \sqrt{n}} \xrightarrow{d} N(0,1)
$$

\subsubsection*{Założenia:}
\begin{enumerate}
    \item Zmienne są niezależne.
    \item Mają ten sam rozkład (i.i.d.).
    \item Istnieje skończona wartość oczekiwana $\mu$ i skończona wariancja $\sigma^2$.
\end{enumerate}

\subsubsection*{Zastosowanie:}
\begin{itemize}
    \item Uzasadnienie stosowania rozkładu normalnego w statystyce.
    \item Budowa przedziałów ufności i testów statystycznych.
    \item Modelowanie błędów pomiaru i zjawisk losowych w naukach przyrodniczych i społecznych.
\end{itemize}

\subsubsection*{Uogólnienia:}
\begin{itemize}
    \item \textbf{Lindeberga i Lyapunowa CTG:} wersje dla zmiennych niezależnych, ale niekoniecznie identycznie rozłożonych.
    \item \textbf{CTG dla rozkładów alfa-stabilnych (gdy wariancja nieskończona):}
    Jeśli zmienne mają ciężkie ogony (np. rozkład Pareto), suma po odpowiednim przeskalowaniu może dążyć nie do rozkładu normalnego, lecz do rozkładu \textbf{alfa-stabilnego}.
    Wtedy:

  $$
  \frac{\sum_{i=1}^n X_i - a_n}{b_n} \xrightarrow{d} S_\alpha(\cdot)
  $$

  gdzie $0 < \alpha < 2$, $S_\alpha$ -- rozkład alfa-stabilny, $b_n \sim n^{1/\alpha}$.
\end{itemize}

\subsection{Pochodna funkcji}

\subsubsection*{Definicja:}
Pochodna funkcji $f$ w punkcie $x_0$ to granica ilorazu różnicowego (jeśli istnieje):

$$
f'(x_0) = \lim_{h \to 0} \frac{f(x_0 + h) - f(x_0)}{h}
$$

\subsubsection*{Interpretacja geometryczna:}
Pochodna to \textbf{nachylenie stycznej} do wykresu funkcji w punkcie $x_0$. Pokazuje, jak szybko zmienia się wartość funkcji.

\subsubsection*{Metody obliczania:}
\begin{itemize}
    \item \textbf{Analityczne:} Reguły różniczkowania (iloczyn, iloraz, złożenie funkcji -- reguła łańcuchowa).
    \item \textbf{Numeryczne:} Przybliżenia różnicowe, np.

  $$
  f'(x) \approx \frac{f(x+h) - f(x)}{h} \quad \text{(różnica prosta)}
  $$

  lub centralna:

  $$
  f'(x) \approx \frac{f(x+h) - f(x-h)}{2h}
  $$
\end{itemize}

\subsection{Całka Riemanna, całki oznaczone i nieoznaczone}

\subsubsection*{Całka Riemanna (konstrukcja):}
Dla funkcji $f$ określonej na przedziale $[a, b]$, całka oznaczona to granica sum Riemanna:

$$
\int_a^b f(x)\,dx = \lim_{\|\mathcal{P}\| \to 0} \sum_{i=1}^n f(c_i) \Delta x_i
$$

gdzie $\mathcal{P}$ to podział przedziału, $c_i \in [x_{i-1}, x_i]$, $\Delta x_i = x_i - x_{i-1}$.

\subsubsection*{Całka oznaczona:}

$$
\int_a^b f(x)\,dx
$$
zwraca liczbę, interpretowana jako pole pod wykresem funkcji między $a$ i $b$.

\subsubsection*{Całka nieoznaczona:}

$$
\int f(x)\,dx
$$
zbiór wszystkich funkcji pierwotnych $F(x)$, takich że $F'(x) = f(x)$.

\subsubsection*{Własności:}
\begin{itemize}
    \item Liniowość: $\int (af + bg) = a \int f + b \int g$
    \item Monotoniczność, addytywność względem przedziału
    \item Jeśli $f$ ciągła na $[a, b]$, to całkowalna
\end{itemize}

\subsubsection*{Podstawowe twierdzenia rachunku całkowego (Newtona-Leibniza):}
\begin{enumerate}
    \item Jeśli $F$ -- funkcja pierwotna $f$, to:

    $$
    \int_a^b f(x)\,dx = F(b) - F(a)
    $$
    \item Jeśli $f$ ciągła, to funkcja:

    $$
    F(x) = \int_a^x f(t)\,dt
    $$

    jest różniczkowalna i $F'(x) = f(x)$
\end{enumerate}


\subsection{Szereg Taylora i Maclaurina}

\subsubsection*{Wzór Taylora:}
Dla funkcji $f$ nieskończenie różniczkowalnej w punkcie $a$:

$$
f(x) = \sum_{n=0}^{\infty} \frac{f^{(n)}(a)}{n!}(x - a)^n
$$

\subsubsection*{Szereg Maclaurina:}
To szczególny przypadek szeregu Taylora dla $a = 0$:

$$
f(x) = \sum_{n=0}^{\infty} \frac{f^{(n)}(0)}{n!}x^n
$$

\subsubsection*{Warunki zbieżności:}
\begin{itemize}
    \item Zbieżność szeregu nie oznacza automatycznie równości z funkcją -- potrzebna \textbf{zbieżność jednostajna} lub spełnienie warunku z resztą:

  $$
  R_n(x) = f(x) - \sum_{k=0}^n \frac{f^{(k)}(a)}{k!}(x - a)^k \to 0 \quad \text{dla } n \to \infty
  $$
    \item Dla wielu funkcji rozwinięcie Taylora zbiega tylko w otoczeniu punktu $a$ (promień zbieżności $R > 0$).
\end{itemize}

\subsubsection*{Zastosowania:}
\begin{itemize}
    \item Aproksymacja funkcji (szczególnie w analizie numerycznej)
    \item Obliczenia przybliżone (np. trygonometryczne, wykładnicze)
    \item Rozwiązywanie równań różniczkowych
    \item Fizyka (np. rozwinięcia wokół stanu równowagi)
\end{itemize}

\subsection{Kryteria zbieżności szeregów}

\subsubsection*{Kryterium porównawcze:}
Jeśli $0 \leq a_n \leq b_n$ dla dużych $n$, i $\sum b_n$ zbieżny, to $\sum a_n$ też zbieżny.
Odwrotnie: jeśli $\sum a_n$ rozbieżny i $a_n \geq b_n \geq 0$, to $\sum b_n$ też rozbieżny.

\subsubsection*{Kryterium d'Alemberta (ilorazowe):}
Jeśli

$$
\lim_{n \to \infty} \left| \frac{a_{n+1}}{a_n} \right| = L
$$

to:
\begin{itemize}
    \item jeśli $L < 1$ $\rightarrow$ szereg zbieżny,
    \item jeśli $L > 1$ lub $= \infty$ $\rightarrow$ rozbieżny,
    \item jeśli $L = 1$ $\rightarrow$ brak informacji (kryterium nie rozstrzyga).
\end{itemize}

\subsubsection*{Kryterium Cauchy'ego (pierwiastkowe):}
Jeśli

$$
\limsup_{n \to \infty} \sqrt[n]{|a_n|} = L,
$$

to:
\begin{itemize}
    \item $L < 1$ $\rightarrow$ szereg zbieżny,
    \item $L > 1$ $\rightarrow$ rozbieżny,
    \item $L = 1$ $\rightarrow$ kryterium nie rozstrzyga.
\end{itemize}

\subsubsection*{Kryterium Leibniza (dla szeregów naprzemiennych):}
Jeśli $a_n$ są dodatnie, monotonicznie malejące i $\lim a_n = 0$, to szereg:

$$
\sum (-1)^n a_n
$$

jest \textbf{zbieżny} (ale niekoniecznie bezwzględnie).

\subsection{Ekstrema funkcji (jednej i wielu zmiennych)}

\subsubsection*{Dla funkcji jednej zmiennej $f(x)$:}
\begin{itemize}
    \item \textbf{Warunek konieczny:} Jeśli $f$ ma ekstremum lokalne w punkcie $x_0$, to $f'(x_0) = 0$ (lub nie istnieje).
    \item \textbf{Warunek wystarczający:}
    \begin{itemize}
        \item Jeśli $f''(x_0) > 0$, to minimum lokalne.
        \item Jeśli $f''(x_0) < 0$, to maksimum lokalne.
        \item Jeśli $f''(x_0) = 0$, potrzebne dalsze badanie (np. wyższe pochodne lub wykres).
    \end{itemize}
\end{itemize}

\subsubsection*{Dla funkcji wielu zmiennych $f(x, y, \ldots)$:}
\begin{itemize}
    \item \textbf{Warunek konieczny:}
    Punkt krytyczny $\nabla f = 0$ (wszystkie pochodne cząstkowe równe 0).
    \item \textbf{Warunek wystarczający (test Hessego):}
    Obliczamy macierz Hessego $H$ (macierz drugich pochodnych).
    \begin{itemize}
        \item Jeśli $H$ dodatnio określona $\rightarrow$ minimum.
        \item Ujemnie określona $\rightarrow$ maksimum.
        \item Nieokreślona $\rightarrow$ punkt siodłowy.
        \item Zdegenerowana $\rightarrow$ test nie rozstrzyga.
    \end{itemize}
\end{itemize}

\subsubsection*{Ekstrema na zbiorach domkniętych:}
Sprawdzamy wartości funkcji:
\begin{itemize}
    \item w punktach krytycznych wewnątrz zbioru,
    \item na brzegu (np. ograniczając funkcję do brzegu i szukając ekstremów),
    \item porównujemy wartości -- największa/minimalna to ekstrema globalne.
\end{itemize}

\subsection{Liczby zespolone}

\subsubsection*{Postać algebraiczna:}

$$
z = a + bi, \quad a, b \in \mathbb{R}, \quad i^2 = -1
$$

\subsubsection*{Postać trygonometryczna:}

$$
z = r(\cos \varphi + i \sin \varphi), \quad r = |z| = \sqrt{a^2 + b^2},\ \varphi = \arg z
$$

\subsubsection*{Postać wykładnicza (Eulera):}

$$
z = re^{i\varphi}, \quad \text{bo } e^{i\varphi} = \cos \varphi + i \sin \varphi
$$

\subsubsection*{Wzór de Moivre'a:}
Dla $z = \cos \varphi + i \sin \varphi$:

$$
z^n = \cos(n\varphi) + i \sin(n\varphi)
$$

Lub ogólniej:

$$
(re^{i\varphi})^n = r^n e^{in\varphi}
$$

\subsection{Transformata Fouriera}

\subsubsection*{Definicja (dla funkcji całkowalnej $f \in L^1(\mathbb{R})$):}

$$
\hat{f}(\xi) = \int_{-\infty}^{\infty} f(x) e^{-2\pi i x \xi} \, dx
$$

\subsubsection*{Własności:}
\begin{itemize}
    \item Liniowość
    \item Przesunięcie: $f(x - a) \leftrightarrow e^{-2\pi i a \xi} \hat{f}(\xi)$
    \item Skalowanie: $f(ax) \leftrightarrow \frac{1}{|a|} \hat{f}\left(\frac{\xi}{a}\right)$
    \item Transformata pochodnej:

    $$
    f'(x) \leftrightarrow 2\pi i \xi \cdot \hat{f}(\xi)
    $$
    \item Transformata funkcji parzystej jest rzeczywista, nieparzystej -- urojona
\end{itemize}

\subsubsection*{Związek z funkcją charakterystyczną:}
Funkcja charakterystyczna zmiennej losowej $X$:

$$
\varphi_X(t) = \mathbb{E}[e^{itX}]
$$

to (z dokładnością do stałej) \textbf{odwrotna transformata Fouriera} rozkładu prawdopodobieństwa $X$.
Jest to analogiczna transformata, ale z $e^{itx}$ zamiast $e^{-2\pi i x \xi}$.


\section*{Rachunek prawdopodobieństwa i procesy stochastyczne}
\addcontentsline{toc}{section}{Rachunek prawdopodobieństwa i procesy stochastyczne}
\subsection{Proces Poissona}
\subsubsection*{Definicja:}
Proces Poissona to losowy proces skokowy $(N(t))_{t \geq 0}$, opisujący liczbę zdarzeń, które zaszły do czasu $t$, gdzie odstępy między kolejnymi zdarzeniami są niezależne i mają rozkład wykładniczy z parametrem $\lambda > 0$.

\subsubsection*{Własności:}

\begin{enumerate}
    \item \textbf{Start w zerze:} $N(0) = 0$.
    \item \textbf{Niezależność przyrostów:} Liczba zdarzeń w niepokrywających się przedziałach czasu jest niezależna.
    \item \textbf{Stacjonarność przyrostów:} Rozkład liczby zdarzeń zależy tylko od długości przedziału czasu.
    \item \textbf{Rozkład:} $P(N(t) = k) = \frac{(\lambda t)^k}{k!} e^{-\lambda t}$, $k = 0,1,2,\dots$
\end{enumerate}

\subsubsection*{Generowanie trajektorii:}
Generujemy kolejne odstępy między zdarzeniami $T_i$ z rozkładu wykładniczego $Exp(\lambda)$, a następnie tworzymy czasy zdarzeń $S_n = T_1 + T_2 + \dots + T_n$. Proces przyjmuje wartość $n$ na przedziale $[S_n, S_{n+1})$.

\subsection{Proces Wienera (Ruch Browna)}
\subsubsection*{Definicja:}
Proces Wienera $(W(t))_{t \geq 0}$ to proces stochastyczny o ciągłych trajektoriach, który spełnia:

\begin{enumerate}
    \item $W(0) = 0$,
    \item niezależność przyrostów,
    \item przyrosty mają rozkład normalny: $W(t+s) - W(s) \sim \mathcal{N}(0, t)$,
    \item trajektorie są ciągłe z prawdopodobieństwem 1.
\end{enumerate}

\subsubsection*{Własności:}

\begin{itemize}
    \item \textbf{Średnia:} $E[W(t)] = 0$,
    \item \textbf{Wariancja:} $\text{Var}[W(t)] = t$,
    \item \textbf{Niezależność i stacjonarność przyrostów},
    \item \textbf{Gaussianowska natura:} każde skończone zbiory wartości mają rozkład normalny.
\end{itemize}

\subsubsection*{Samopodobieństwo:}
Dla dowolnej stałej $a > 0$, proces $W(at)$ ma taki sam rozkład jak $\sqrt{a} W(t)$. To oznacza, że proces jest \textbf{samopodobny rzędu $H = 1/2$}.


\subsection{Prawa Wielkich Liczb (PWL): Słabe i Mocne}
Prawa Wielkich Liczb opisują warunki, w których średnia arytmetyczna ciągu niezależnych zmiennych losowych zbiega do wartości oczekiwanej.

\subsubsection*{Słabe Prawo Wielkich Liczb (SPWL):}

Jeśli $X_1, X_2, \dots$ są niezależne, identycznie rozłożone (i.i.d.) z $E[X_i] = \mu$, to

$$
\overline{X}_n = \frac{1}{n} \sum_{i=1}^n X_i \xrightarrow{P} \mu.
$$

(Zbieżność \textbf{w prawdopodobieństwie}.)

\subsubsection*{Mocne Prawo Wielkich Liczb (MPWL):}

Dla tego samego ciągu:

$$
\overline{X}_n \xrightarrow{a.s.} \mu.
$$

(Zbieżność \textbf{prawie na pewno}.)

\subsubsection*{Rodzaje zbieżności zmiennych losowych:}

\begin{enumerate}
    \item \textbf{Zbieżność prawie na pewno (a.s.):} $P(\lim_{n \to \infty} X_n = X) = 1$,
    \item \textbf{Zbieżność w prawdopodobieństwie:} $\forall \varepsilon > 0, \ \lim_{n \to \infty} P(|X_n - X| > \varepsilon) = 0$,
    \item \textbf{Zbieżność w rozkładzie (dystrybuancji):} $F_{X_n}(x) \to F_X(x)$ dla punktów ciągłości $F_X$,
    \item \textbf{Zbieżność w średniej kwadratowej:} $\lim_{n \to \infty} E[(X_n - X)^2] = 0$.
\end{enumerate}

Zależności:

$$
\text{Zb. a.s.} \Rightarrow \text{zb. w prawdopodobieństwie} \Rightarrow \text{zb. w rozkładzie}.
$$

\subsection{Metoda Monte Carlo}
\subsubsection*{Idea:}
Metoda Monte Carlo polega na użyciu losowania (symulacji) do przybliżenia wartości liczbowych, np. całek, wartości oczekiwanych, rozwiązań równań itp.

\subsubsection*{Przykład:}
Aby oszacować wartość oczekiwaną $E[f(X)]$, generujemy $X_1, ..., X_n$ i obliczamy

$$
\hat{\mu}_n = \frac{1}{n} \sum_{i=1}^n f(X_i).
$$

Dla dużego $n$, $\hat{\mu}_n \approx E[f(X)]$.

\subsubsection*{Podstawa teoretyczna:}
\textbf{Prawa Wielkich Liczb} -- gwarantują, że $\hat{\mu}_n \to E[f(X)]$ przy rosnącym $n$, co uzasadnia poprawność metody.

\subsubsection*{Zalety:}
\begin{itemize}
    \item Łatwość implementacji,
    \item Możliwość zastosowania w problemach wysokowymiarowych lub trudnych analitycznie.
\end{itemize}

\subsection{Stacjonarność procesu stochastycznego (w węższym i szerszym sensie)}
\subsubsection*{Stacjonarność w węższym sensie (tzw. stacjonarność drugiego rzędu):}
Proces $(X_t)_{t \in T}$ jest stacjonarny w węższym sensie, jeśli:

\begin{enumerate}
    \item $E[X_t] = \mu = \text{const}$,
    \item $\text{Cov}(X_t, X_{t+h}) = \gamma(h)$ -- zależy tylko od przesunięcia $h$, a nie od $t$.
\end{enumerate}

\subsubsection*{Stacjonarność w szerszym sensie (ściśle stacjonarny):}
Proces $(X_t)$ jest stacjonarny w szerszym sensie, jeśli rozkład $(X_{t_1}, ..., X_{t_k})$ jest taki sam jak $(X_{t_1+h}, ..., X_{t_k+h})$ dla dowolnych $t_1,...,t_k$ i $h$.
(Inaczej: cały rozkład jest niezmienniczy na przesunięcia czasu.)

\subsubsection*{Przykłady:}

\begin{itemize}
    \item \textbf{Proces stacjonarny w sensie szerokim:} proces autoregresyjny AR(1) z $|\phi| < 1$.
    \item \textbf{Proces ściśle stacjonarny:} proces o niezależnych i identycznie rozłożonych zmiennych $X_t \sim \text{Exp}(\lambda)$.
    \item \textbf{Proces Wienera}: \textbf{nie jest stacjonarny} -- wariancja rośnie z czasem.
\end{itemize}

\subsection{Funkcja charakterystyczna}
\subsubsection*{Definicja:}
Funkcja charakterystyczna zmiennej losowej $X$ to funkcja $\varphi_X(t) = E[e^{itX}]$, $t \in \mathbb{R}$.

\subsubsection*{Własności:}

\begin{enumerate}
    \item Zawsze istnieje (również dla zmiennych bez momentów),
    \item $\varphi_X(0) = 1$,
    \item $|\varphi_X(t)| \leq 1$,
    \item Funkcja charakterystyczna określa jednoznacznie rozkład zmiennej losowej,
    \item $\varphi_{aX + b}(t) = e^{itb} \varphi_X(at)$,
    \item Dla niezależnych $X, Y$: $\varphi_{X+Y}(t) = \varphi_X(t) \cdot \varphi_Y(t)$.
\end{enumerate}

\subsubsection*{Zastosowanie:}

\begin{itemize}
    \item Identyfikacja rozkładu,
    \item Dowód twierdzenia centralnego granicznego (CLT),
    \item Badanie zbieżności w rozkładzie,
    \item Ułatwia obliczenia przy sumach niezależnych zmiennych.
\end{itemize}

\subsection{Martyngały}
\subsubsection*{Definicja:}
Proces $(X_n)_{n \geq 0}$ to \textbf{martyngał} względem filtracji $(\mathcal{F}_n)$, jeśli:

\begin{enumerate}
    \item $X_n$ jest $\mathcal{F}_n$-mierzalny,
    \item $E[|X_n|] < \infty$,
    \item $E[X_{n+1} \mid \mathcal{F}_n] = X_n$ prawie na pewno.
\end{enumerate}

\subsubsection*{Intuicja:}
Brak przewagi w grze -- przyszła wartość średnia równa jest obecnej, biorąc pod uwagę dostępną informację.

\subsubsection*{Przykłady:}

\begin{itemize}
    \item $X_n = S_n = \sum_{i=1}^n \xi_i$, gdzie $\xi_i$ są niezależne, mają wartość oczekiwaną 0.
    \item $X_n = E[Y \mid \mathcal{F}_n]$, gdzie $Y$ ma skończoną wartość oczekiwaną.
    \item Proces $W(t)$ -- proces Wienera -- jest martyngałem względem swojej naturalnej filtracji.
\end{itemize}


\section*{Statystyka}
\addcontentsline{toc}{section}{Statystyka}
\subsection{Metody estymacji parametrów}
\subsubsection*{Metoda największej wiarygodności (MLE)}
Polega na wyznaczeniu takich wartości parametrów rozkładu, które maksymalizują funkcję wiarygodności -- czyli prawdopodobieństwo zaobserwowania danej próby. Rozwiązanie uzyskuje się zwykle poprzez różniczkowanie logarytmu funkcji wiarygodności i rozwiązanie układu równań.

\subsubsection*{Metoda momentów}
Polega na przyrównaniu teoretycznych momentów rozkładu (np. wartości oczekiwanej, wariancji) do odpowiadających im momentów empirycznych wyznaczonych z próby. Liczbę równań dobiera się do liczby estymowanych parametrów.

\subsection{Regresja liniowa}
\subsubsection*{Model regresji liniowej}
Model opisuje zależność zmiennej zależnej $Y$ od jednej lub więcej zmiennych niezależnych $X$:
$Y = \beta_0 + \beta_1 X_1 + \ldots + \beta_p X_p + \varepsilon$
gdzie $\varepsilon$ to składnik losowy.

\subsubsection*{Założenia modelu klasycznego (Gaussa-Markowa):}

\begin{itemize}
    \item Liniowość modelu względem parametrów
    \item $E(\varepsilon) = 0$
    \item $\text{Var}(\varepsilon) = \sigma^2$, stała wariancja (homoskedastyczność)
    \item Brak autokorelacji składników losowych
    \item Niezależność obserwacji
    \item Zmienna losowa $\varepsilon$ ma rozkład normalny (dla wnioskowania)
\end{itemize}

\subsubsection*{Estymacja parametrów -- metoda najmniejszych kwadratów (MNK)}
Parametry modelu estymuje się przez minimalizację sumy kwadratów reszt:
$\min_\beta \sum (Y_i - \hat{Y}_i)^2$
Rozwiązanie analityczne:
$\hat{\beta} = (X^T X)^{-1} X^T Y$

\subsection{Testowanie hipotez statystycznych}
\subsubsection*{Błąd I rodzaju}
Popełniamy go, gdy odrzucamy hipotezę zerową $H_0$, mimo że jest prawdziwa. Prawdopodobieństwo tego błędu to poziom istotności $\alpha$.

\subsubsection*{Błąd II rodzaju}
Popełniamy go, gdy nie odrzucamy $H_0$, mimo że hipoteza alternatywna $H_1$ jest prawdziwa. Prawdopodobieństwo tego błędu to $\beta$.

\subsubsection*{p-wartość}
To najmniejsze możliwe $\alpha$, dla którego odrzucilibyśmy $H_0$ przy zaobserwowanej statystyce testowej. Jeżeli p-wartość < $\alpha$, odrzucamy $H_0$.

\subsubsection*{Test Kołmogorowa-Smirnowa (K-S)}
To nieparametryczny test zgodności, porównujący dystrybuantę empiryczną z dystrybuantą teoretyczną (jednowymiarowy przypadek). Statystyka testowa to maksymalna wartość bezwzględna różnicy między tymi funkcjami. Służy np. do sprawdzania normalności.


\section*{Algebra}
\addcontentsline{toc}{section}{Algebra}
\subsection{Wyznacznik macierzy}
\subsubsection*{Definicja:}
Wyznacznik to liczba przypisana macierzy kwadratowej, oznaczana np. $\det(A)$, wykorzystywana m.in. do badania odwracalności macierzy. Macierz jest odwracalna wtedy i tylko wtedy, gdy jej wyznacznik jest różny od zera.

\subsubsection*{Metody obliczania:}
\begin{itemize}
    \item \textbf{Rozwinięcie Laplace'a:} rozwinięcie względem dowolnego wiersza lub kolumny przy użyciu mniejszych wyznaczników (minorów). Dobrze nadaje się dla małych macierzy.
    \item \textbf{Eliminacja Gaussa:} przekształcamy macierz do postaci trójkątnej; wyznacznik to iloczyn elementów na przekątnej, z uwzględnieniem znaków wynikających z zamian wierszy.
\end{itemize}

\subsubsection*{Interpretacja geometryczna:}
Dla macierzy $2 \times 2$ lub $3 \times 3$, wyznacznik odpowiada odpowiednio polu powierzchni lub objętości równoległoboku/równoległościanu rozpiętego przez kolumny macierzy. Znak wyznacznika informuje o orientacji układu (np. dodatni -- zachowana orientacja).

\subsection{Wartości i wektory własne}
\subsubsection*{Definicja:}
Dla macierzy kwadratowej $A$, liczba $\lambda$ jest \textbf{wartością własną}, jeśli istnieje niezerowy wektor $v$, taki że:

$$
A v = \lambda v
$$

Wektor $v$ nazywany jest \textbf{wektorem własnym} odpowiadającym wartości $\lambda$.

\subsubsection*{Zastosowania:}
\begin{itemize}
    \item \textbf{Diagonalizacja macierzy:} jeśli macierz $A$ ma $n$ liniowo niezależnych wektorów własnych, to można ją przedstawić w postaci $A = PDP^{-1}$, gdzie $D$ to macierz diagonalna z wartościami własnymi. Ułatwia to np. potęgowanie macierzy.
    \item \textbf{Zastosowania praktyczne:} analiza stabilności układów dynamicznych, PCA (analiza głównych składowych), mechanika kwantowa, przetwarzanie obrazów, grafy.
\end{itemize}

\subsection{Układy równań liniowych}
\subsubsection*{Liczba rozwiązań:}
Układ równań liniowych może mieć:
\begin{itemize}
    \item jedno rozwiązanie (układ oznaczony),
    \item nieskończenie wiele rozwiązań (układ nieoznaczony),
    \item brak rozwiązań (układ sprzeczny).
\end{itemize}

\subsubsection*{Twierdzenie Kroneckera-Capellego:}
Niech $\text{rz}(A)$ oznacza rząd macierzy głównej, a $\text{rz}(A|b)$ rząd macierzy rozszerzonej.
\begin{itemize}
    \item Jeśli $\text{rz}(A) \neq \text{rz}(A|b)$, to układ jest \textbf{sprzeczny} (brak rozwiązań).
    \item Jeśli $\text{rz}(A) = \text{rz}(A|b) = r$, to układ jest niesprzeczny. Wtedy, dla $n$ niewiadomych:
    \begin{itemize}
        \item Jeśli $r = n$, układ jest \textbf{oznaczony} (jedno rozwiązanie).
        \item Jeśli $r < n$, układ jest \textbf{nieoznaczony} (nieskończenie wiele rozwiązań).
    \end{itemize}
\end{itemize}


\section*{Metody numeryczne}
\addcontentsline{toc}{section}{Metody numeryczne}
\subsection{Metody numerycznego rozwiązywania równań różniczkowych}
\subsubsection*{Metoda Eulera}
To najprostsza metoda numeryczna dla równań postaci $y' = f(x, y)$, z danym warunkiem początkowym $y(x_0) = y_0$. Przybliżenie wyznacza się iteracyjnie:
$y_{n+1} = y_n + h f(x_n, y_n)$
gdzie $h$ to krok siatki. Metoda ma rząd dokładności 1 i może być niestabilna przy większym kroku.

\subsubsection*{Metody Rungego-Kutty}
To rodzina dokładniejszych metod; najczęściej stosowana to metoda RK4 (czwartego rzędu):
$y_{n+1} = y_n + \frac{h}{6}(k_1 + 2k_2 + 2k_3 + k_4)$
gdzie $k_1, k_2, k_3, k_4$ to przybliżenia nachyleń w różnych punktach przedziału. RK4 zapewnia wysoką dokładność przy umiarkowanym koszcie obliczeniowym.

\subsection{Metody znajdowania miejsc zerowych funkcji}
\subsubsection*{Metoda bisekcji}
Polega na iteracyjnym dzieleniu przedziału $[a, b]$, w którym funkcja zmienia znak (czyli $f(a) \cdot f(b) < 0$). W każdej iteracji wybiera się połowę przedziału, w której występuje zmiana znaku. Metoda jest wolna (zbieżność liniowa), ale zawsze zbieżna.

\subsubsection*{Metoda Newtona (Newtona-Raphsona)}
Stosuje wzór:
$x_{n+1} = x_n - \frac{f(x_n)}{f'(x_n)}$
Wymaga znajomości pochodnej $f'(x)$ i odpowiedniego punktu startowego. Zbieżność jest kwadratowa, ale może zawieść, jeśli $f'(x_n) \approx 0$ lub punkt startowy jest źle dobrany.

\subsection{Interpolacja wielomianowa}
\subsubsection*{Idea}
Interpolacja polega na znalezieniu wielomianu $P_n(x)$, który przechodzi przez dane punkty $(x_i, y_i)$. Przykładowe metody to Lagrange'a lub Newtona. Celem jest aproksymacja funkcji znanej tylko w punktach.

\subsubsection*{Efekt Rungego}
Przy interpolacji wielomianowej na równomiernej siatce, szczególnie dla funkcji o dużych zmianach (np. $f(x) = \frac{1}{1 + x^2}$), wielomian może silnie oscylować na końcach przedziału. Zjawisko to nazywa się efektem Rungego i wskazuje, że zwiększanie stopnia wielomianu nie zawsze poprawia jakość interpolacji.


\end{document}
