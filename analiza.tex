\subsection{Centralne Twierdzenie Graniczne (CTG)}

\subsubsection*{Definicja:}
Centralne Twierdzenie Graniczne mówi, że jeśli $X_1, X_2, ..., X_n$ są niezależnymi, identycznie rozłożonymi zmiennymi losowymi o skończonej wartości oczekiwanej $\mu$ i skończonej wariancji $\sigma^2$, to suma (lub średnia) tych zmiennych po odpowiednim przeskalowaniu dąży rozkładem do rozkładu normalnego, gdy $n \to \infty$.

Formalnie (dla sumy):

$$
\frac{\sum_{i=1}^n X_i - n\mu}{\sigma \sqrt{n}} \xrightarrow{d} N(0,1)
$$

Równoważnie dla średniej arytmetycznej $\bar{X}_n = \frac{1}{n}\sum_{i=1}^n X_i$:
$$
\frac{\bar{X}_n - \mu}{\sigma / \sqrt{n}} \xrightarrow{d} N(0,1)
$$

\subsubsection*{Założenia:}
\begin{enumerate}
    \item Zmienne są niezależne.
    \item Mają ten sam rozkład (i.i.d.).
    \item Istnieje skończona wartość oczekiwana $\mu$ i skończona wariancja $\sigma^2$.
\end{enumerate}

\subsubsection*{Zastosowanie:}
\begin{itemize}
    \item Uzasadnienie stosowania rozkładu normalnego w statystyce.
    \item Budowa przedziałów ufności i testów statystycznych.
    \item Modelowanie błędów pomiaru i zjawisk losowych w naukach przyrodniczych i społecznych.
\end{itemize}

\subsubsection*{Uogólnienia:}
\begin{itemize}
    \item \textbf{Lindeberga i Lyapunowa CTG:} wersje dla zmiennych niezależnych, ale niekoniecznie identycznie rozłożonych.
    \item \textbf{CTG dla rozkładów alfa-stabilnych (gdy wariancja nieskończona):}
    Jeśli zmienne mają ciężkie ogony (np. rozkład Pareto), suma po odpowiednim przeskalowaniu może dążyć nie do rozkładu normalnego, lecz do rozkładu \textbf{alfa-stabilnego}.
    Wtedy:

  $$
  \frac{\sum_{i=1}^n X_i - a_n}{b_n} \xrightarrow{d} S_\alpha(\cdot)
  $$

  gdzie $0 < \alpha < 2$, $S_\alpha$ -- rozkład alfa-stabilny, $b_n \sim n^{1/\alpha}$.
\end{itemize}

\subsection{Pochodna funkcji}

\subsubsection*{Definicja:}
Pochodna funkcji $f$ w punkcie $x_0$ to granica ilorazu różnicowego (jeśli istnieje):

$$
f'(x_0) = \lim_{h \to 0} \frac{f(x_0 + h) - f(x_0)}{h}
$$

\subsubsection*{Interpretacja geometryczna:}
Pochodna to \textbf{nachylenie stycznej} do wykresu funkcji w punkcie $x_0$. Pokazuje, jak szybko zmienia się wartość funkcji.

\subsubsection*{Metody obliczania:}
\begin{itemize}
    \item \textbf{Analityczne:} Reguły różniczkowania (iloczyn, iloraz, złożenie funkcji -- reguła łańcuchowa).
    \item \textbf{Numeryczne:} Przybliżenia różnicowe, np.

  $$
  f'(x) \approx \frac{f(x+h) - f(x)}{h} \quad \text{(różnica prosta)}
  $$

  lub centralna:

  $$
  f'(x) \approx \frac{f(x+h) - f(x-h)}{2h}
  $$
\end{itemize}

\subsection{Całka Riemanna, całki oznaczone i nieoznaczone}

\subsubsection*{Całka Riemanna (konstrukcja):}
Dla funkcji $f$ określonej na przedziale $[a, b]$, całka oznaczona to granica sum Riemanna:

$$
\int_a^b f(x)\,dx = \lim_{\|\mathcal{P}\| \to 0} \sum_{i=1}^n f(c_i) \Delta x_i
$$

gdzie $\mathcal{P}$ to podział przedziału, $c_i \in [x_{i-1}, x_i]$, $\Delta x_i = x_i - x_{i-1}$.

\subsubsection*{Całka oznaczona:}

$$
\int_a^b f(x)\,dx
$$
zwraca liczbę, interpretowana jako pole pod wykresem funkcji między $a$ i $b$.

\subsubsection*{Całka nieoznaczona:}

$$
\int f(x)\,dx
$$
zbiór wszystkich funkcji pierwotnych $F(x)$, takich że $F'(x) = f(x)$.

\subsubsection*{Własności:}
\begin{itemize}
    \item Liniowość: $\int (af + bg) = a \int f + b \int g$
    \item Monotoniczność, addytywność względem przedziału
    \item Jeśli $f$ ciągła na $[a, b]$, to całkowalna
\end{itemize}

\subsubsection*{Podstawowe twierdzenia rachunku całkowego (Newtona-Leibniza):}
\begin{enumerate}
    \item Jeśli $F$ -- funkcja pierwotna $f$, to:

    $$
    \int_a^b f(x)\,dx = F(b) - F(a)
    $$
    \item Jeśli $f$ ciągła, to funkcja:

    $$
    F(x) = \int_a^x f(t)\,dt
    $$

    jest różniczkowalna i $F'(x) = f(x)$
\end{enumerate}


\subsection{Szereg Taylora i Maclaurina}

\subsubsection*{Wzór Taylora:}
Dla funkcji $f$ nieskończenie różniczkowalnej w otoczeniu punktu $a$. Poniższa równość zachodzi, gdy reszta $R_n(x)$ we wzorze Taylora dąży do zera:

$$
f(x) = \sum_{n=0}^{\infty} \frac{f^{(n)}(a)}{n!}(x - a)^n
$$

\subsubsection*{Szereg Maclaurina:}
To szczególny przypadek szeregu Taylora dla $a = 0$:

$$
f(x) = \sum_{n=0}^{\infty} \frac{f^{(n)}(0)}{n!}x^n
$$

\subsubsection*{Warunki zbieżności:}
\begin{itemize}
    \item Zbieżność szeregu nie oznacza automatycznie równości z funkcją -- potrzebna \textbf{zbieżność jednostajna} lub spełnienie warunku z resztą:

  $$
  R_n(x) = f(x) - \sum_{k=0}^n \frac{f^{(k)}(a)}{k!}(x - a)^k \to 0 \quad \text{dla } n \to \infty
  $$
    \item Dla wielu funkcji rozwinięcie Taylora zbiega tylko w otoczeniu punktu $a$ (promień zbieżności $R > 0$).
\end{itemize}

\subsubsection*{Zastosowania:}
\begin{itemize}
    \item Aproksymacja funkcji (szczególnie w analizie numerycznej)
    \item Obliczenia przybliżone (np. trygonometryczne, wykładnicze)
    \item Rozwiązywanie równań różniczkowych
    \item Fizyka (np. rozwinięcia wokół stanu równowagi)
\end{itemize}

\subsection{Kryteria zbieżności szeregów}

\subsubsection*{Kryterium porównawcze:}
Jeśli $0 \leq a_n \leq b_n$ dla dużych $n$, i $\sum b_n$ zbieżny, to $\sum a_n$ też zbieżny.
Odwrotnie: jeśli $0 \le b_n \le a_n$ i $\sum b_n$ rozbieżny, to $\sum a_n$ też rozbieżny.

\subsubsection*{Kryterium d'Alemberta (ilorazowe):}
Jeśli

$$
\lim_{n \to \infty} \left| \frac{a_{n+1}}{a_n} \right| = L
$$

to:
\begin{itemize}
    \item jeśli $L < 1$ $\rightarrow$ szereg zbieżny,
    \item jeśli $L > 1$ lub $= \infty$ $\rightarrow$ rozbieżny,
    \item jeśli $L = 1$ $\rightarrow$ brak informacji (kryterium nie rozstrzyga).
\end{itemize}

\subsubsection*{Kryterium Cauchy'ego (pierwiastkowe):}
Jeśli

$$
\limsup_{n \to \infty} \sqrt[n]{|a_n|} = L,
$$

to:
\begin{itemize}
    \item $L < 1$ $\rightarrow$ szereg zbieżny,
    \item $L > 1$ $\rightarrow$ rozbieżny,
    \item $L = 1$ $\rightarrow$ kryterium nie rozstrzyga.
\end{itemize}

\subsubsection*{Kryterium Leibniza (dla szeregów naprzemiennych):}
Jeśli $a_n$ są dodatnie, monotonicznie malejące i $\lim a_n = 0$, to szereg:

$$
\sum (-1)^n a_n
$$

jest \textbf{zbieżny} (ale niekoniecznie bezwzględnie).

\subsection{Ekstrema funkcji (jednej i wielu zmiennych)}

\subsubsection*{Dla funkcji jednej zmiennej $f(x)$:}
\begin{itemize}
    \item \textbf{Warunek konieczny:} Jeśli $f$ ma ekstremum lokalne w punkcie $x_0$, to $f'(x_0) = 0$ (lub nie istnieje).
    \item \textbf{Warunek wystarczający:}
    \begin{itemize}
        \item Jeśli $f''(x_0) > 0$, to minimum lokalne.
        \item Jeśli $f''(x_0) < 0$, to maksimum lokalne.
        \item Jeśli $f''(x_0) = 0$, potrzebne dalsze badanie (np. wyższe pochodne lub wykres).
    \end{itemize}
\end{itemize}

\subsubsection*{Dla funkcji wielu zmiennych $f(x, y, \ldots)$:}
\begin{itemize}
    \item \textbf{Warunek konieczny:}
    Punkt krytyczny $\nabla f = 0$ (wszystkie pochodne cząstkowe równe 0).
    \item \textbf{Warunek wystarczający (test Hessego):}
    Obliczamy macierz Hessego $H$ (macierz drugich pochodnych).
    \begin{itemize}
        \item Jeśli $H$ dodatnio określona $\rightarrow$ minimum.
        \item Ujemnie określona $\rightarrow$ maksimum.
        \item Nieokreślona $\rightarrow$ punkt siodłowy.
        \item Zdegenerowana $\rightarrow$ test nie rozstrzyga.
    \end{itemize}
\end{itemize}

\subsubsection*{Ekstrema na zbiorach domkniętych:}
Sprawdzamy wartości funkcji:
\begin{itemize}
    \item w punktach krytycznych wewnątrz zbioru,
    \item na brzegu (np. ograniczając funkcję do brzegu i szukając ekstremów),
    \item porównujemy wartości -- największa/minimalna to ekstrema globalne.
\end{itemize}

\subsection{Liczby zespolone}

\subsubsection*{Postać algebraiczna:}

$$
z = a + bi, \quad a, b \in \mathbb{R}, \quad i^2 = -1
$$

\subsubsection*{Postać trygonometryczna:}

$$
z = r(\cos \varphi + i \sin \varphi), \quad r = |z| = \sqrt{a^2 + b^2},\ \varphi = \arg z
$$

\subsubsection*{Postać wykładnicza (Eulera):}

$$
z = re^{i\varphi}, \quad \text{bo } e^{i\varphi} = \cos \varphi + i \sin \varphi
$$

\subsubsection*{Wzór de Moivre'a:}
Dla $z = r\left(\cos \varphi + i \sin \varphi\right)$:

$$
z^n = |z|^n\left(\cos(n\varphi) + i \sin(n\varphi)\right)
$$

\subsection{Transformata Fouriera}

\subsubsection*{Definicja (dla funkcji całkowalnej $f \in L^1(\mathbb{R})$):}

$$
\hat{f}(\xi) = \int_{-\infty}^{\infty} f(x) e^{-2\pi i x \xi} \, dx
$$

\subsubsection*{Własności:}
\begin{itemize}
    \item Liniowość
    \item Przesunięcie: $f(x - a) \leftrightarrow e^{-2\pi i a \xi} \hat{f}(\xi)$
    \item Skalowanie: $f(ax) \leftrightarrow \frac{1}{|a|} \hat{f}\left(\frac{\xi}{a}\right)$
    \item Transformata pochodnej:

    $$
    f'(x) \leftrightarrow 2\pi i \xi \cdot \hat{f}(\xi)
    $$
    \item Transformata funkcji parzystej jest rzeczywista, nieparzystej -- urojona
\end{itemize}

\subsubsection*{Związek z funkcją charakterystyczną:}
Funkcja charakterystyczna zmiennej losowej $X$:

$$
\varphi_X(t) = \mathbb{E}[e^{itX}]
$$

Funkcja charakterystyczna $\varphi_X(t)$ jest transformatą Fouriera gęstości prawdopodobieństwa zmiennej losowej $X$,
ale z inną konwencją znaku w wykładniku ($e^{itx}$ zamiast $e^{-2\pi i x \xi}$) i~skalowania.
